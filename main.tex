\documentclass[12pt]{report}

% =====================
% PAQUETES BÁSICOS
% =====================
\usepackage{blindtext}                           % Para generar texto de ejemplo
\usepackage[T1]{fontenc}                         % Codificación de fuente
\usepackage{fontspec}                            % Cambiar el tipo de letra
\usepackage[a4paper, margin=25.4mm]{geometry}    % Configuración de la hoja
\usepackage{graphicx}                            % Insertar gráficos e imágenes
\usepackage[font=small,labelfont=bf]{caption}    % Configuración de captions para tablas e imágenes
\usepackage{subcaption}                          % Subcaptions para imágenes
\usepackage{hyperref}                            % Hacer links clicables
\usepackage{listings}                            % Mostrar código de programación
\usepackage{amsmath}                             % Para símbolos y ecuaciones matemáticas
\usepackage{xcolor}                              % Para manejar colores
\usepackage{indentfirst}                         % Identación en el primer párrafo de cada sección
\usepackage{multirow}                            % Tablas con celdas de múltiples filas
\usepackage{titlesec}                            % Control sobre títulos de sección y capítulo
\usepackage{array}                               % Mejorar tablas
\usepackage{enumitem}                            % Mejorar listas
\usepackage{polyglossia}                         % Para manejar idiomas (como español)
\usepackage[titles]{tocloft}                     % Personalizar tabla de contenidos

% =====================
% CONFIGURACIÓN DE IDIOMA Y FUENTE
% =====================
\setdefaultlanguage{spanish}                     % Configura el idioma a español
\setmainfont{Times New Roman}                    % Cambia la fuente principal

% =====================
% CONFIGURACIÓN DE TABLAS Y FIGURAS
% =====================
\usepackage{chngcntr}
\counterwithout{figure}{chapter}                % Evita que las figuras estén numeradas por capítulo
\counterwithout{table}{chapter}                 % Evita que las tablas estén numeradas por capítulo

% Para cambiar "Cuadro" por "Tabla"
\gappto\captionsspanish{\renewcommand{\tablename}{Tabla}} 

% =====================
% CONFIGURACIÓN DE LISTINGS (Código de Programación)
% =====================
\lstset{
  language=Python,                              % Cambia esto al lenguaje de tu preferencia
  basicstyle=\ttfamily\footnotesize,
  keywordstyle=\color{blue},
  commentstyle=\color{green},
  stringstyle=\color{red},
  numbers=left,
  numberstyle=\tiny\color{gray},
  stepnumber=1,
  numbersep=5pt,
  backgroundcolor=\color{white},
  showspaces=false,
  showstringspaces=false,
  showtabs=false,
  frame=single,
  tabsize=2
}

% =====================
% CONFIGURACIÓN DE TABLE OF CONTENTS (TOC)
% =====================
\renewcommand{\contentsname}{\centerline{\Large\bfseries Tabla de Contenido}}
\renewcommand{\cftchapfont}{\normalfont}
\renewcommand{\cftchappagefont}{\normalfont}
\renewcommand{\cftsecfont}{\normalfont}
\renewcommand{\cftsecpagefont}{\normalfont}
\renewcommand{\cftsubsecfont}{\normalfont}
\renewcommand{\cftsubsecpagefont}{\normalfont}
\renewcommand{\cftchapleader}{\cftdotfill{\cftdotsep}}
\renewcommand{\cftchappresnum}{Capítulo }
\renewcommand{\cftchapaftersnum}{.}
\setlength{\cftchapnumwidth}{5em}
\renewcommand{\cftfigpresnum}{Figura }
\renewcommand{\cftfigaftersnum}{.}
\setlength{\cftfignumwidth}{4.5em}
\renewcommand{\cfttabpresnum}{Tabla }  
\renewcommand{\cfttabaftersnum}{.}
\setlength{\cfttabnumwidth}{4em}              

% =====================
% CONFIGURACIÓN DE TITLESEC (Títulos)
% =====================
\titleformat{\chapter}
  {\normalfont\Large\bfseries}
  {Capítulo \thechapter.}{1em}{}
\titlespacing*{\chapter}{0pt}{-20pt}{20pt}

% =====================
% CONFIGURACIÓN DE BIBLIOGRAFÍA
% =====================
\usepackage[backend=biber, style=apa, sorting=nyt]{biblatex}
\addbibresource{references.bib}                 % Añade el archivo de referencias

% =====================
% CONFIGURACIÓN DE FANCYHDR (Encabezado y pie de página)
% =====================
\usepackage{fancyhdr}
\pagestyle{fancy}
\fancyhead{}                                    
\setlength{\headheight}{5.1pt}                  
\renewcommand{\headrulewidth}{0pt}              

% =====================
% DOCUMENTO
% =====================
\begin{document}

% =====================
% PORTADA
% =====================
\begin{titlepage}
  \centering
  {\bfseries\fontsize{14pt}{28pt}\selectfont PONTIFICIA UNIVERSIDAD CATÓLICA DEL PERÚ \par}
  {\bfseries\fontsize{14pt}{28pt}\selectfont FACULTAD DE CIENCIAS E INGENIERÍA \par}
  \vfill
  {\includegraphics[width=1\textwidth]{Images/logo_pucp.png}\par} % Cambia la ruta al logo si es necesario
  {\bfseries\fontsize{16pt}{28pt}\selectfont Nombre del proyecto de fin de carrera \par}
  {\bfseries\fontsize{14pt}{28pt}\selectfont Tesis para obtener el título de Ingeniero Informático \par}
  \vfill
  {\bfseries\fontsize{14pt}{28pt}\selectfont AUTORA: \par}
  {\fontsize{14pt}{28pt}\selectfont Nombre del Autor \par}
  \vfill
  {\bfseries\fontsize{14pt}{28pt}\selectfont ASESORES: \par}
  {\fontsize{14pt}{28pt}\selectfont Asesor 1 \\ Asesor 2 \par}
  \vfill
  {\fontsize{14pt}{28pt}\selectfont Lima, Septiembre, 2024 \par}
\end{titlepage}

% =====================
% CONFIGURACIÓN GENERAL DEL DOCUMENTO
% =====================
\pagenumbering{roman}                           % Números romanos para prefacios
\setlength{\parindent}{1cm}                     % Sangría de 1 cm
\fontsize{12}{24}\selectfont                    % Tamaño de letra 12 con interlineado 24

% =====================
% ÍNDICES
% =====================
\renewcommand{\contentsname}{\centerline{\Large\bfseries Tabla de Contenido}}
\tableofcontents
\clearpage

\renewcommand{\listfigurename}{\centerline{\Large\bfseries Índice de Figuras}}
\listoffigures
\clearpage

\renewcommand{\listtablename}{\centerline{\Large\bfseries Índice de Tablas}}
\listoftables
\clearpage

% =====================
% CAPÍTULOS PRINCIPALES (Aquí incluirás los capítulos de tu documento)
% =====================
\pagenumbering{arabic}                          % Números arábigos para los capítulos
\chapter{Generalidades}\label{c:Generalidades} % Título del capítulo y etiqueta para referencia
Este capítulo presenta las generalidades del trabajo, incluyendo la problemática, objetivos y métodos empleados. Este capítulo presenta las generalidades del trabajo, incluyendo la problemática, objetivos y métodos empleados. 

% =====================
% SECCIÓN: PROBLEMÁTICA
% =====================
\section{Problemática}
Este capítulo presenta las generalidades del trabajo, incluyendo la problemática, objetivos y métodos empleados. Este capítulo presenta las generalidades del trabajo, incluyendo la problemática, objetivos y métodos empleados. Este capítulo presenta las generalidades del trabajo, incluyendo la problemática, objetivos y métodos empleados.

\subsection{Árbol de problemas}

% Ejemplo de inclusión de una figura
\begin{figure}[htbp]
    \centering
    \includegraphics[width=1\textwidth]{Images/logo_pucp.png} % Ruta de la imagen a incluir
    \caption{Árbol de problemas} % Título de la figura
    \label{fig:arbol_problemas} % Etiqueta para referenciar la figura en el documento
\end{figure}

\subsection{Descripción}

% Introduce aquí la descripción de la problemática
La problemática del estudio está basada en ejemplo de cita \parencite{petticrew_roberts_2006}. 
Este capítulo presenta las generalidades del trabajo, incluyendo la problemática, objetivos y métodos empleados. Este capítulo presenta las generalidades del trabajo, incluyendo la problemática, objetivos y métodos empleados. Este capítulo presenta las generalidades del trabajo, incluyendo la problemática, objetivos y métodos empleados.
% Ejemplo de cita en formato APA

\subsection{Problema Seleccionado}
% Describe el problema específico que fue seleccionado para abordar en el trabajo
Este capítulo presenta las generalidades del trabajo, incluyendo la problemática, objetivos y métodos empleados. Este capítulo presenta las generalidades del trabajo, incluyendo la problemática, objetivos y métodos empleados. Este capítulo presenta las generalidades del trabajo, incluyendo la problemática, objetivos y métodos empleados.

% =====================
% SECCIÓN: OBJETIVOS
% =====================
\section{Objetivos}

\subsection{Objetivo General}
% Introduce el objetivo principal de tu trabajo
Este capítulo presenta las generalidades del trabajo, incluyendo la problemática, objetivos y métodos empleados. Este capítulo presenta las generalidades del trabajo, incluyendo la problemática, objetivos y métodos empleados. Este capítulo presenta las generalidades del trabajo, incluyendo la problemática, objetivos y métodos empleados.


\subsection{Objetivos Específicos}
\begin{enumerate}[label=O\arabic*.] % Lista numerada para objetivos específicos con el formato "O1.", "O2.", etc.
    \item ... % Primer objetivo específico
    \item ... % Segundo objetivo específico
    \item ... % Tercer objetivo específico
\end{enumerate}

% =====================
% SECCIÓN: MÉTODOS Y PROCEDIMIENTOS
% =====================
\section{Métodos y procedimientos}
Este capítulo presenta las generalidades del trabajo, incluyendo la problemática, objetivos y métodos empleados. Este capítulo presenta las generalidades del trabajo, incluyendo la problemática, objetivos y métodos empleados. Este capítulo presenta las generalidades del trabajo, incluyendo la problemática, objetivos y métodos empleados.
% Describe aquí los métodos y procedimientos que se utilizaron para llevar a cabo la investigación o el proyecto
                % Problemática y objetivos
\chapter{Marco Legal/Regulatorio/Conceptual/otros} \label{c::Marcos}

\section{Introducción}
En este capítulo Presentación del marco. Presentación del marco.

\section{Desarrollo del marco conceptual}
Presentación del marco. Presentación del marco. Presentación del marco. Presentación del marco. Presentación del marco. Presentación del marco.

\subsection{Concepto}
Presentación del marco. Presentación del marco. Presentación del marco. Presentación del marco. Presentación del marco. Presentación del marco.

\section{Desarrollo del marco legal}
Presentación del marco. Presentación del marco. Presentación del marco. Presentación del marco. Presentación del marco. Presentación del marco.

\subsection{Concepto}
Presentación del marco. Presentación del marco. Presentación del marco. Presentación del marco. Presentación del marco. Presentación del marco.
                   % Marcos
\chapter{Estado del Arte}\label{ch::estadoArte} % Título del capítulo y etiqueta para referencia
Sunt fugit corrupti fugiat necessitatibus officia. Corrupti amet consequuntur. Voluptatibus aut amet possimus assumenda eos necessitatibus.

% =====================
% SECCIÓN: OBJETIVOS DE REVISIÓN
% =====================
\section{Objetivos de revisión}
Para iniciar con la revisión sistemática se plantearon los siguientes objetivos:
\begin{itemize}
    \item O1: Ratione explicabo praesentium aliquid error.
    \item O2: Iusto voluptatibus nihil temporibus fuga tenetur.
    \item O3: Tempore iure pariatur quia voluptates in ad odit officia animi sit.
\end{itemize}

% =====================
% SECCIÓN: PREGUNTAS DE REVISIÓN
% =====================
\section{Preguntas de revisión}
Aspernatur aspernatur accusantium dolorum dolore recusandae. Repellat nesciunt possimus voluptas eveniet.

% =====================
% TABLA DE CRITERIOS PICOC
% =====================
\begin{table}[htbp]
    \caption{Criterios PICOC}
    \centering
    \begin{tabular}{|c|p{10cm}|}
        \hline
        \textbf{Criterio} & \textbf{Descripción} \\ \hline
        Población & Amet officiis fugit. \\ \hline
        Intervención & Quia quos nemo. \\ \hline
        Comparación & Nulla quaerat laboriosam dolores. \\ \hline
        Salidas & Ab consequuntur a molestiae ullam eos vitae. \\ \hline
        Contexto & Pariatur minus odio occaecati itaque. \\ \hline
    \end{tabular}
\end{table}

% =====================
% SECCIÓN: ESTRATEGIA DE BÚSQUEDA
% =====================
\section{Estrategia de búsqueda}
Numquam hic amet fugit. Sed a a error. Expedita quae error perferendis cum.

\subsection{Motores de búsqueda a usar}
Los siguientes motores de búsqueda fueron considerados para la revisión sistemática:
\begin{itemize}
    \item IEEE Xplore
    \item Google Scholar
    \item ACM Digital Library
\end{itemize}

\subsection{Cadenas de búsqueda a usar}
Laboriosam aperiam perspiciatis perferendis quod repellendus. Animi repellat ipsum fugiat deleniti est.

\subsection{Documentos encontrados}
Recusandae aut quo occaecati laborum iure dolore itaque. Occaecati consequatur voluptatibus autem iste. In ratione quos praesentium.

% =====================
% CRITERIOS DE INCLUSIÓN Y EXCLUSIÓN
% =====================
\subsection{Criterios de inclusión y exclusión}
\subsubsection*{Criterios de inclusión}
\begin{enumerate}[label=CI.\arabic*.]
    \item Quaerat ducimus occaecati culpa delectus quibusdam impedit incidunt excepturi.
    \item Architecto vel sapiente assumenda recusandae.
\end{enumerate}

\subsubsection*{Criterios de exclusión}
\begin{enumerate}[label=CE.\arabic*.]
    \item Officia id iure beatae similique.
    \item Labore eos facilis et laborum quod natus ipsam.
\end{enumerate}

% =====================
% FORMULARIO DE EXTRACCIÓN DE DATOS
% =====================
\section{Formulario de extracción de datos}
Expedita impedit optio rerum minima voluptatibus quam molestias. Quisquam dicta dolore similique optio. Sunt fugit nostrum iusto.

% =====================
% RESULTADOS DE REVISIÓN
% =====================
\section{Resultados de Revisión}
Hic architecto tenetur nihil consequuntur cumque sapiente. Temporibus cum magnam laborum vero quasi nostrum.

% =====================
% DISCUSIÓN
% =====================
\section{Discusión}
\subsection{Respuesta a la pregunta 1}
A continuación, se responderá la primera pregunta de investigación: \textbf{\textit{``¿Pregunta?''}}\newline
Fugit consequuntur earum vero.

\subsection{Respuesta a la pregunta 2}
A continuación, se responderá la segunda pregunta de investigación: \textbf{\textit{``¿Pregunta?''}}\newline
Nemo molestias minima rem porro sit. Incidunt quasi beatae ipsam vitae magnam doloribus soluta.

\subsection{Respuesta a la pregunta 3}
A continuación, se responderá la tercera pregunta de investigación: \textbf{\textit{``¿Pregunta?''}}\newline
Est doloremque molestias beatae.

\subsection{Respuesta a la pregunta 4}
A continuación, se responderá la cuarta pregunta de investigación: \textbf{\textit{``¿Pregunta?''}}\newline
Reiciendis voluptatibus nihil eligendi. Labore eius vel saepe doloremque.

% =====================
% CONCLUSIONES
% =====================
\section{Conclusiones}
Culpa asperiores illo similique. At omnis natus laudantium dolor illum veniam laudantium.
               % Estado del Arte

% =====================
% BIBLIOGRAFÍA
% =====================
\addcontentsline{toc}{chapter}{Referencias}
\renewcommand*{\bibfont}{\fontsize{10}{10}\selectfont} 
\printbibliography

% =====================
% APÉNDICES (Descomentar si se incluyen apéndices)
% \appendix
% \include{Chapters/4Appendix}

\end{document}
